\documentclass{article}
\usepackage{times,amsmath,amsthm,amsfonts,eucal,graphicx}
\usepackage{subcaption}


\setlength{\oddsidemargin}{0.25 in}
\setlength{\evensidemargin}{-0.25 in}
\setlength{\topmargin}{-0.6 in}
\setlength{\textwidth}{6.5 in}
\setlength{\textheight}{8.5 in}
\setlength{\headsep}{0.75 in}
\setlength{\parindent}{0 in}
\setlength{\parskip}{0.1 in}

%
% The following commands set up the lecnum (lecture number)
% counter and make various numbering schemes work relative
% to the lecture number.
%
\newcounter{lecnum}
\renewcommand{\thepage}{\thelecnum-\arabic{page}}
\renewcommand{\thesection}{\thelecnum.\arabic{section}}
\renewcommand{\theequation}{\thelecnum.\arabic{equation}}
\renewcommand{\thefigure}{\thelecnum.\arabic{figure}}
\renewcommand{\thetable}{\thelecnum.\arabic{table}}

%
% A few symbols that we will be using often in this course.
\newcommand{\indep}{{\bot\negthickspace\negthickspace\bot}}
\newcommand{\notindep}{{\not\negthickspace\negthinspace{\bot\negthickspace\negthickspace\bot}}}
\newcommand{\definedtobe}{\stackrel{\Delta}{=}}
\renewcommand{\choose}[2]{{{#1}\atopwithdelims(){#2}}}
\newcommand{\argmax}[1]{{\hbox{$\underset{#1}{\mbox{argmax}}\;$}}}
\newcommand{\argmin}[1]{{\hbox{$\underset{#1}{\mbox{argmin}}\;$}}}

%
% The following macro is used to generate the header.
%
\newcommand{\lecture}[4]{
   \pagestyle{myheadings}
   \thispagestyle{plain}
   \newpage
   \setcounter{lecnum}{#1}
   \setcounter{page}{1}
   \noindent
   \begin{center}
   \framebox{
      \vbox{\vspace{2mm}
    \hbox to 6.58in { {\bf [CS200]-STNT-2
                        \hfill Fall 2020-21} }
    \hbox to 6.58in { {\bf 
                        \hfill } }
       \vspace{4mm}
       \hbox to 6.28in { {\Large \hfill Homework: 03 \hfill} }
       \vspace{5mm}
       \hbox to 2.28in { {\textbf{16. Technomakers:} 
       		    $\hspace*{0.01mm}$ Anupam Kumar (11940160), Abdur Rahman Khan (11940020),
       		$\hspace*{0.1mm}$Ruchit Prakash Saxena (11941040)} }
      \vspace{1mm}
  \hbox to 1.0in { {    } }
  
}
   }
   \end{center}
   
   \vspace*{3cm}
}

%
% Convention for citations is authors' initials followed by the year.
% For example, to cite a paper by Leighton and Maggs you would type
% \cite{LM89}, and to cite a paper by Strassen you would type \cite{S69}.
% (To avoid bibliography problems, for now we redefine the \cite command.)
% Also commands that create a suitable format for the reference list.
\renewcommand{\cite}[1]{[#1]}
\def\beginrefs{\begin{list}%
        {[\arabic{equation}]}{\usecounter{equation}
         \setlength{\leftmargin}{2.0truecm}\setlength{\labelsep}{0.4truecm}%
         \setlength{\labelwidth}{1.6truecm}}}
\def\endrefs{\end{list}}
\def\bibentry#1{\item[\hbox{[#1]}]}

%Use this command for a figure; it puts a figure in wherever you want it.
%usage: \fig{NUMBER}{CAPTION}{.eps FILE TO INCLUDE}{WIDTH-IN-INCHES}
\newcommand{\fig}[4]{
			\begin{center}
	                \includegraphics[width=#4,clip=true]{#3} \\
			Figure \thelecnum.#1:~#2
			\end{center}
	}

\theoremstyle{definition}
% Use these for theorems, lemmas, proofs, etc.
\newtheorem{theorem}{Theorem}[lecnum]
\newtheorem{lemma}[theorem]{Lemma}
\newtheorem{proposition}[theorem]{Proposition}
\newtheorem{claim}[theorem]{Claim}
\newtheorem{corollary}[theorem]{Corollary}
\newtheorem{definition}[theorem]{Definition}
\newtheorem{problem}{Solution of problem}

% \newenvironment{proof}{{\bf Proof:}}{\hfill\rule{2mm}{2mm}}

% **** IF YOU WANT TO DEFINE ADDITIONAL MACROS FOR YOURSELF, PUT THEM HERE:

\begin{document}
%FILL IN THE RIGHT INFO.
%\lecture{**LECTURE-NUMBER**}{**DATE**}{**LECTURER**}{**SCRIBE**}
\lecture{1}{}{xxx}
%\footnotetext{These notes are partially based on those of Nigel Mansell.}

% **** YOUR NOTES GO HERE
\textbf{2. (a)}\\
The git commands are as follows : \\\\
$ \$ $ git branch test\\
$ \$ $ git commit\\
$ \$ $ git checkout test\\
$ \$ $ git commit\\
$ \$ $ git checkout master\\
$ \$ $ git merge test\\\\
Now the given git graph is formed shown in \textbf{figure 1.1}\\\\\\
$ \$ $ git checkout HEAD\^{}\\\\\\
The graph after doing this is shown in \textbf{figure 1.2}\\\\\\


\begin{figure}
	
	\begin{center}
		\includegraphics[scale=0.6]{HW3_q2.0.jpg} \caption{After merging test}
	\end{center}
	
	
\end{figure}

\begin{figure}
	
	\begin{center}
		\includegraphics[scale=0.51]{HW3_q2.1.jpg} \caption{After checking out HEAD to one commit before}
	\end{center}
\end{figure}

When we execute the git command \textbf{$ \$ $ git checkout HEAD\^{}} then,\\
to resove the ambiguity , git makes use of SHA value or hash value of the merge commit which contains information about the branch which is merged with the other branch hence ,
since we checked out \textbf{ master} before merging with \textbf{test} so, after merging, git stores the commit hash value b2cd7bd.. as the parent commit(into which the other branch is merged) and also stores the hash value of the commit of \textbf{test} branch i.e., clf9449.. as the commit of branch which has been merged .\\
Hence, git resolves the ambiguity by checking the hash value of the merge commit i.e, 85931fe.. \\\\\\\\\\\\\\\\\\\\\\\\\\

\textbf{2. (b)}\\\\
The git commands are as follows : \\\\
$ \$ $ git branch test\\
$ \$ $ git commit\\
$ \$ $ git checkout test\\
$ \$ $ git commit\\
$ \$ $ git merge master\\\\
Now the given git graph is formed shown in \textbf{figure 1.3}\\\\\\
$ \$ $ git checkout HEAD\^{}\\\\\\
The graph after doing this is shown in \textbf{figure 1.4}\\\\\\
When we execute the git command \textbf{$ \$ $ git checkout HEAD\^{}} then,\\
to resove the ambiguity , git makes use of SHA value or hash value of the merge commit which contains information about the branch which is merged with the other branch hence ,
since we checked out \textbf{test} before merging with \textbf{master} so, after merging, git stores the commit hash value be30631..  as the parent commit(into which the other branch is merged) and also stores the hash value of the commit of \textbf{master} branch i.e., efb53d5.. as the commit of branch which has been merged .\\
Hence, git resolves the ambiguity by checking the hash value of the merge commit i.e, bd112fd.. \\\\\\\\\\\\\\\\\\\\

\begin{figure}
	
	\begin{center}
		\includegraphics[scale=0.6]{HW3_q2.2.jpg} \caption{After merging master}
	\end{center}
	
	
\end{figure}

\begin{figure}
	
	\begin{center}
		\includegraphics[scale=0.51]{HW3_q2.3.jpg} \caption{After checking out HEAD to one commit before}
	\end{center}
\end{figure}
\end{document}
